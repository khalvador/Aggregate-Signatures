%%%%%%%%%%%%%%%%%%%%%%%%%%%%%%%%%%%%%%%%%%%%%%%%%%%%%%%%%%%%%%%%%%%%%%%%%%%%%%%%
% FLAGS
%
\newif\iffull
%\fulltrue
\fullfalse
%%%%%%%%%%%%%%%%%%%%%%%%%%%%%%%%%%%%%%%%%%%%%%%%%%%%%%%%%%%%%%%%%%%%%%%%%%%%%%%%


%%%%%%%%%%%%%%%%%%%%%%%%%%%%%%%%%%%%%%%%%%%%%%%%%%%%%%%%%%%%%%%%%%%%%%%%%%%%%%%%
\iffull
\documentclass[11pt,letterpaper]{article}
%\usepackage[notes=true,later=true]{dtrt}
\usepackage{fullpage}
\else
%\documentclass[orivec,envcountsect,envcountsame]{llncs}
\documentclass[orivec,envcountsect,envcountsame,11pt]{llncs}
\usepackage[notes=xxx,llncssubsub]{dtrt}
\usepackage{fullpage}
\pagestyle{plain} 
\fi

\iffull
\usepackage[backend=bibtex8,style=alphabetic,maxnames=8,maxalphanames=6,sorting=anyt]{biblatex}
% https://cryptobib.di.ens.fr/
\bibliography{local.bib,abbrev3.bib,crypto_crossref.bib,}
\else
\usepackage[
style=alphabetic,
firstinits,
backend=bibtex,
maxalphanames=4,
minalphanames=3,
maxbibnames=15,
]{biblatex}


% \DeclareNameFormat{}{\usebibmacro{name:first-last}{}{#5}{#1}{#7}\usebibmacro{name:andothers}}
%\DeclareFieldFormat[misc]{title}{\mkbibquote{#1\MLInpSizdot}}
\bibliography{references}
\fi
\renewcommand*{\bibfont}{\small}
%%%%%%%%%%%%%%%%%%%%%%%%%%%%%%%%%%%%%%%%%%%%%%%%%%%%%%%%%%%%%%%%%%%%%%%%%%%%%%%%



%%%%%%%%%%%%%%%%%%%%%%%%%%%%%%%%%%%%%%%%%%%%%%%%%%%%%%%%%%%%%%%%%%%%%%%%%%%%%%%%
% PACKAGES
\iffull
\usepackage{amsthm}
\else
\let\proof\relax
\let\endproof\relax
\usepackage{amsthm}
\fi
\usepackage{amssymb,amsfonts,amsmath}
\usepackage{bbm}
%\usepackage{times}
\usepackage{microtype}
\usepackage{tikz}
\usetikzlibrary{matrix}
\usetikzlibrary{shapes,arrows}
\usepackage{graphicx}
\usepackage{verbatim}
\usepackage{array}
\usepackage{multirow}
\usepackage{latexsym}
\usepackage{paralist}
\usepackage{enumitem}
\setlist[itemize]{leftmargin=*}
\setlist[enumerate]{leftmargin=*}
\usepackage[capitalise]{cleveref}
\crefname{step}{Step}{Steps}
\usepackage{caption}
\usepackage{dsfont}
\usepackage{url}
\usepackage{braket}
\usepackage{mathrsfs}
\usepackage{color}
\usepackage{soul}
\usepackage{mdframed}
\usepackage{mathtools}
\usepackage{bm}
\usepackage{bbm}
\usepackage{setspace}
\usepackage{multirow}
\usepackage{adjustbox}
\usepackage{booktabs}
\usepackage{algorithm}
\usepackage{algpseudocode}
\usepackage{xspace}
\usepackage{graphics}
\usepackage[notions,operators,sets,keys,ff,adversary,primitives,complexity,asymptotics,lambda,landau,advantage]{cryptocode}
\iffull
\usepackage[margin=4mm,small,labelfont=bf]{caption}
\fi
%% the following is commaon for all examples in mdframed manual
\mdfsetup{skipabove=\topskip,skipbelow=\topskip}
%%% upto here
\newcounter{theo}[section]
%%%%%%%%%%%%%%%%%%%%%%%%%%%%%%%%%%%%%%%%%%%%%%%%%%%%%%%%%%%%%%%%%%%%%%%%%%%%%%%%


%%%%%%%%%%%%%%%%%%%%%%%%%%%%%%%%%%%%%%%%%%%%%%%%%%%%%%%%%%%%%%%%%%%%%%%%%%%%%%%%
% THEOREM ENVIRONMENTS
\newtheorem{idefinition}{Definition}%[section]
\newtheorem{itheorem}{Theorem}%[section]
\newtheorem{icorollary}{Corollary}%[section]
\newtheorem{ilemma}{Lemma}%[section]
\iffull
\newtheorem{theorem}{Theorem}[section]
\newtheorem{corollary}[theorem]{Corollary}
\newtheorem{proposition}[theorem]{Proposition}
\newtheorem{definition}[theorem]{Definition}
\newtheorem{lemma}[theorem]{Lemma}
\newtheorem{claim}[theorem]{Claim}
\newtheorem{observation}[theorem]{Observation}
\newtheorem{fact}[theorem]{Fact}
\newtheorem{question}[theorem]{Question}
\fi

\theoremstyle{definition} % non-italics
\iffull
\newtheorem{example}{Example}
\newtheorem{remark}[theorem]{Remark}
\fi
\newtheorem{protocol}[theorem]{Protocol}
\newtheorem{construction}[theorem]{Construction}
%%%%%%%%%%%%%%%%%%%%%%%%%%%%%%%%%%%%%%%%%%%%%%%%%%%%%%%%%%%%%%%%%%%%%%%%%%%%%%%%



%%%%%%%%%%%%%%%%%%%%%%%%%%%%%%%%%%%%%%%%%%%%%%%%%%%%%%%%%%%%%%%%%%%%%%%%%%%%%%%%
% NOTES
\newcommand{\ale}[1]{\dtcolornote[Ale]{red}{#1}}
\newcommand{\giacomo}[1]{\dtcolornote[Giacomo]{blue}{#1}}
\newcommand{\khanh}[1]{\dtcolornote[Khanh]{magenta}{#1}}
\newcommand{\martin}[1]{\dtcolornote[Martin]{purple}{#1}}
\newcommand{\sasha}[1]{\dtcolornote[Sasha]{orange}{#1}}

\newcommand{\BCGdiff}[1]{\textcolor{blue}{#1}}
\newcommand{\DEdiff}[1]{\color{blue}{#1}}

%%%%%%%%%%%%%%%%%%%%%%%%%%%%%%%%%%%%%%%%%%%%%%%%%%%%%%%%%%%%%%%%%%%%%%%%%%%%%%%%



%%%%%%%%%%%%%%%%%%%%%%%%%%%%%%%%%%%%%%%%%%%%%%%%%%%%%%%%%%%%%%%%%%%%%%%%%%%%%%%%
% FORMATTING
\allowdisplaybreaks
\newcommand{\FormatAuthor}[3]{
	\begin{tabular}{c}
		#1 \\ {\small\texttt{#2}} \\ {\small #3}
	\end{tabular}
}
\newcommand{\doclearpage}{%
	\iffull
	\clearpage
	\fi
}
\newcommand{\DoQuote}[1]{``#1''}
\newcommand{\defemph}[1]{\textbf{\emph{#1}}}
\newcommand{\keywords}[1]{\bigskip\par\noindent{\footnotesize\textbf{Keywords\/}: #1}}
%%%%%%%%%%%%%%%%%%%%%%%%%%%%%%%%%%%%%%%%%%%%%%%%%%%%%%%%%%%%%%%%%%%%%%%%%%%%%%%%

% tikz

% Define block styles
\tikzstyle{source} = [rectangle, draw, fill=gray!20, 
text width=10em, text centered, minimum height=8em]

\tikzstyle{protocol} = [rectangle, draw, fill=white!20, 
text width=10em, text centered, minimum height=8em]

\tikzstyle{transformation} = [rectangle, draw, fill=white!20, 
text width=8em, text centered, rounded corners, minimum height=6em]

\tikzstyle{line} = [draw, -latex']

%%%%%%%%%%%%%%%%%%%%%%%%%%%%%%%%%%%%%%%%%%%%%%%%%%%%%%%%%%%%%%%%%%%%%%%%%%%%%%%%
% MACROS
\newcommand{\cmark}{\ding{51}}%
\newcommand{\xmark}{\ding{55}}%\\
%===Editorial===


%Camera version variable, if not camera version, comment the line
%\def\camera{}

\newcommand{\fitpage}[1]{
	\resizebox{\textwidth}{!}{\ensuremath{#1}}
}

%=====Arrows=====
\newcommand{\verylongleftarrow}[1]{
	\setlength{\unitlength}{.01in}
	\begin{picture}(#1,1) \put(#1,0){\vector(-1,0){#1}} \end{picture}
}

\newcommand{\verylongrightarrow}[1]{
	\setlength{\unitlength}{.01in}
	\begin{picture}(#1,1) \put(0,0){\vector(1,0){#1}} \end{picture}
}

\newcommand{\leftgoing}[1]{{\stackrel{{\displaystyle #1}} {\verylongleftarrow{85}}}}
\newcommand{\rightgoing}[2]{{\stackrel{{\displaystyle #1}} {\verylongrightarrow{85}}}}
%\newcommand{\fitpage}[1]{\resizebox{\textwidth}{!}{\ensuremath{#1}}}
%===Notation===
\newcommand{\ulp}{\mathsf{ulp}}
\newcommand{\arp}{\mathsf{arp}}
\newcommand\scalemath[2]{\scalebox{#1}{\mbox{\ensuremath{\displaystyle #2}}}}
\newcommand{\gsign}{\mathsf{GSign}}
\newcommand{\gverify}{\mathsf{GVerify}}
\newcommand{\gsdec}{\mathsf{GSdec}}
\newcommand{\gmsk}{\mathsf{gmsk}}
\newcommand{\pid}{\mathfrak{p}}
\newcommand{\secpam}{\mathsf{N}}
\newcommand{\garb}{\mathsf{garb}}
\newcommand{\msis}{\mathsf{MSIS}}
\newcommand{\mlwe}{\mathsf{MLWE}}
\newcommand{\ro}{\mathsf{H}}
\newcommand{\ntru}{\mathsf{NTRU}}
\newcommand{\bphi}{\boldsymbol{\phi}}
\newcommand{\balpha}{\boldsymbol{\alpha}}
\newcommand{\bgamma}{\boldsymbol{\gamma}}
\newcommand{\nbofeqs}{\rho}
\newcommand{\nbofeqsapprox}{\upsilon_d}
\newcommand{\nbofeqsexact}{\upsilon_e}
\newcommand{\nbofeqseval}{\rho_\mathsf{eval}}
\newcommand{\tr}{\mathsf{Tr}}
\newcommand{\inprodfunction}{\mathsf{T}}
\newcommand{\trace}{\mathrm{Tr}}
\newcommand{\Span}{\mathrm{Span}}
\newcommand{\embed}{\varphi}
\newcommand{\coeffembed}{\mathsf{vec}}
\newcommand{\rank}{\mathsf{rank}}
\newcommand{\ringbasis}{\mathsf{Ring}\text{-}\mathsf{BASIS}}
\newcommand{\basis}{\textsf{BASIS}}
\newcommand{\prisis}{\textsf{PRISIS}}
\newcommand{\powerBasis}{\mathsf{PowerBASIS}}
\newcommand{\GL}{\mathsf{GL}}
\newcommand{\samp}{\mathsf{Samp}}
\newcommand{\aux}{\mathsf{aux}}
\renewcommand{\vec}[1]{\ensuremath{\mathbf{#1}}\xspace}
\newcommand{\vecalpha}{\bm{\alpha}}
\newcommand{\vecnu}{\bm{\nu}}
\newcommand{\vecgamma}{\bm{\gamma}}
\providecommand{\mat}[1]{\ensuremath{\vec{#1}}\xspace}
\providecommand{\ring}[0]{\ensuremath{\mathcal{R}}\xspace}

\newenvironment{theo}[1][]{%
\stepcounter{theo}%
\ifstrempty{#1}%
 {\mdfsetup{%
   frametitle={%
    \tikz[baseline=(current bounding box.east),outer sep=0pt]
    \node[anchor=east,rectangle,fill=blue!20]
         {\strut Construction~\thetheo};}}
 }%
{\mdfsetup{%
  frametitle={%
   \tikz[baseline=(current bounding box.east),outer sep=0pt]
   \node[anchor=east,rectangle,fill=blue!20]
        {\strut ~#1};}}%
 }%
\mdfsetup{innertopmargin=10pt,linecolor=blue!20,%
       linewidth=2pt,topline=true,
       frametitleaboveskip=\dimexpr-\ht\strutbox\relax,}
   \begin{mdframed}[]\relax%
}
{\end{mdframed}}

%%%%%%%%%%%%%%%%%%%%
% Macro generation %
%%%%%%%%%%%%%%%%%%%%
\usepackage{etoolbox}
\def\do#1{\csdef{b#1}{\ensuremath{\mathbb{#1}}}}
\docsvlist{A,B,C,D,E,F,G,H,I,J,K,L,M,N,O,P,Q,R,S,T,U,V,W,X,Y,Z}

\def\do#1{\csdef{c#1}{\ensuremath{\mathcal{#1}}}}
\docsvlist{A,B,C,D,E,F,G,H,I,J,K,L,M,N,O,P,Q,R,S,T,U,V,W,X,Y,Z}

\def\do#1{\csdef{s#1}{\ensuremath{\mathsf{#1}}}}
\docsvlist{A,B,C,D,E,F,G,H,I,J,K,L,M,N,O,P,Q,R,S,T,U,V,W,X,Y,Z}

\def\do#1{\csdef{mat#1}{\ensuremath{\mathbf{#1}}}}
\docsvlist{A,B,C,D,E,F,G,H,I,J,K,L,M,N,O,P,Q,R,S,T,U,V,W,X,Y,Z}

\def\do#1{\csdef{vec#1}{\ensuremath{\mathbf{#1}}}}
\docsvlist{a,b,c,d,e,f,g,h,i,j,k,l,m,n,o,p,q,r,s,t,u,v,w,x,y,z,sk,pk}


%===Adversaries
\newcommand{\Adversary}{\mathcal{A}}
\newcommand{\AltAdversary}{\mathcal{B}}


%Relations
\newcommand{\recRelation}{\mathsf{R}}
\newcommand{\Index}{\mathbbm{i}}
\newcommand{\Instance}{\mathbbm{x}}
\newcommand{\Witness}{\mathbbm{w}}

\newcommand{\FirstRound}{\mathbbm{a}}
\newcommand{\SecondRound}{\mathbbm{c}}
\newcommand{\ThirdRound}{\mathbbm{z}}

%Pseudocode
\makeatletter
\newcommand{\algrule}[1][.2pt]{\par\vskip.5\baselineskip\hrule height #1\par\vskip.5\baselineskip}
\makeatother
\newcommand{\true}{\textbf{true}}
\newcommand{\false}{\textbf{false}}
\newcommand{\abort}{\textbf{abort}}
\newcommand{\gcom}[1]{\hfill $\sslash$#1}
\newcommand{\bool}[1]{\llbracket #1 \rrbracket}
\newcommand{\getsunif}{\overset{\$}{\leftarrow}}
\newcommand{\SecParam}{\lambda}

%===Matrix Commitment
\newcommand{\q}{{\tilde{q}}}
\newcommand{\matrixsetup}{\mathsf{MSetup}}
\newcommand{\matrixcrs}{\mathsf{crs}}
\newcommand{\matrixcommit}{\mathsf{MCommit}}
\newcommand{\matrixstate}{\mathsf{st}}
\newcommand{\matrixeval}{\mathsf{MEval}}
\newcommand{\matrixverify}{\mathsf{MVerify}}


%===Many-to-one commitment
\newcommand{\commitment}{\mathsf{MtO}}
\newcommand{\setup}{\mathsf{Setup}}
\newcommand{\commit}{\mathsf{Commit}}
%\newcommand{\keygen}{\mathsf{KeyGen}}
\newcommand{\open}{\mathsf{Open}}
\newcommand{\recopen}{\mathsf{LevOpen}}
\newcommand{\arity}{\kappa}
\newcommand{\BindingSet}{\mathcal{X}}
\newcommand{\OpenSpace}{\Space_O}
\newcommand{\OpenValidSpace}{\Space_V}
\newcommand{\OpenHonestSpace}{\Space_H}
\newcommand{\Space}{\mathcal{S}}
\newcommand{\BindMessSpace}{\mathcal{S}}
\newcommand{\comkey}{\mathsf{ck}}
\newcommand{\slack}{\omega}

\newcommand{\Message}{\mathsf{m}}
\newcommand{\Opening}{\mathsf{o}}
\newcommand{\veclength}{\ell}
\newcommand{\Decompose}{\mathsf{Decompose}}
\newcommand{\ann}{\text{Ann}}

\newcommand{\idx}{\veci}
\newcommand{\MessSpace}{\Space_M}
\newcommand{\PublicParam}{\mathsf{pp}}
%===Vector Commitment
\newcommand{\veccom}{\mathsf{VC}}
\newcommand{\veccomsetup}{\mathsf{Setup}}
\newcommand{\veccomcommit}{\mathsf{Com}}
\newcommand{\veccomstate}{\mathsf{st}}
\newcommand{\veccomopen}{\mathsf{Open}}
\newcommand{\veccomcheck}{\mathsf{Check}}

%===Polynomial Commitment
\newcommand{\commitmentScheme}{\mathsf{CM}}
\newcommand{\polysetup}{\mathsf{Setup}}
\newcommand{\polycrs}{\mathsf{crs}}
\newcommand{\polyproof}{\pi}
\newcommand{\polycommit}{\mathsf{Commit}}
\newcommand{\polystate}{\mathsf{st}}
\newcommand{\polyeval}{\mathsf{Eval}}
\newcommand{\polyverify}{\mathsf{Verify}}
\newcommand{\polyopen}{\mathsf{Open}}
\newcommand{\polyaggregate}{\mathsf{Aggregate}}
\newcommand{\polyaggverify}{\mathsf{AggVerify}}
\newcommand{\polyProver}{\prov}
\newcommand{\polyVerifier}{\ver}
\newcommand{\polyExtractor}{\mathcal{E}}
\newcommand{\polyChallengeSpace}{\mathcal{C}}

\newcommand{\polydegree}{d}
\newcommand{\crs}{\mathsf{crs}}
\newcommand{\varx}{\mathsf{X}}
\newcommand{\vary}{\mathsf{Y}}
\newcommand{\vari}{\mathsf{I}}
\newcommand{\varj}{\mathsf{J}}
\newcommand{\even}{\mathsf{even}}
\newcommand{\odd}{\mathsf{odd}}
\newcommand{\HammingWt}{\operatorname{w}}
%==Trapdoor Sampling
\newcommand{\trapgen}{\mathsf{TrapGen}}
\newcommand{\genlin}{\mathsf{GL}}
\newcommand{\samplepre}{\mathsf{SamplePre}}
%===Sets and Spaces===
\newcommand{\NumberField}{\mathcal{K}}
\newcommand{\NumberFieldR}{\NumberField_{\RR}}
\newcommand{\Zqt}{\Z_{q^\tau}}
\newcommand{\Fqdl}{\mathbb{F}_{q^{d/l}}}
\newcommand{\Mq}{\mathcal{M}_q}
\newcommand{\R}{\mathcal{R}}
\newcommand{\F}{\mathcal{F}}
\newcommand{\Rq}{\R_q}
\newcommand{\Rp}{\R_p}
\newcommand{\mbin}{m_{\text{bin}}}
\newcommand{\bit}{\{0,1\}}
\newcommand{\Lattice}{\Lambda}
\newcommand{\genLattice}{\mathcal{L}}
\newcommand{\coeff}{\mathsf{coeff}}
\newcommand{\discGauss}{\mathcal{D}}
\newcommand{\smoothing}{\eta}
\newcommand{\minDistance}{\lambda}
\newcommand{\ringInteger}{\mathcal{O}}
\newcommand{\integer}{\mathsf{int}}
%===Matrix===
\makeatletter
\renewcommand*\env@matrix[1][*\c@MaxMatrixCols c]{%
	\hskip -\arraycolsep
	\let\@ifnextchar\new@ifnextchar
	\array{#1}}
\makeatother
\newcommand{\veczero}{\mathbf{0}}
\newcommand{\matzero}{\mathbf{0}}
\newcommand{\matpi}{\mathbf{\Pi}}



%===Norms===
\newcommand{\twonorm}[1]{\left\lVert{#1}\right\rVert_2}
\newcommand{\infnorm}[1]{\left\lVert{#1}\right\rVert_\infty}

%===Lattices===
\newcommand{\polvec}[1]{\bm{#1}}
\newcommand{\hatpol}[1]{\hat{\pol{#1}}}
\newcommand{\ntt}[1]{\mathsf{NTT}\left({#1}\right)}
\newcommand{\nttinv}[1]{\mathsf{NTT}^{-1}\left({#1}\right)}
\newcommand{\aut}{\mathsf{Aut}(\R_q)}
\newcommand{\rot}{\mathsf{rot}}
\newcommand{\inprod}[2]{\langle #1 , #2 \rangle}
\newcommand{\fc}[1]{\widetilde{#1}}
\newcommand{\Degree}{d}
\newcommand{\NTRUTrapGen}{\mathsf{NTRU.TrapGen}}
\newcommand{\NTRUbasis}{\matT_\ntru}
\newcommand{\GaussianSampler}{\mathsf{NTRU.SamplePre}}
%===Zero Knowledge===
\newcommand{\prov}{\mathcal{P}}
\newcommand{\ver}{\mathcal{V}}
\newcommand{\simul}{\mathcal{S}}
\newcommand{\user}{\mathcal{U}}
\newcommand{\Proof}{\pi}
\newcommand{\Extractor}{\polyExtractor}

\newcommand{\rej}[1]{\mathsf{Rej}\left({#1}\right)}
\newcommand{\chal}{\mathcal{C}}
\newcommand{\eqtest}{\overset{?}{=}}

\newcommand{\uu}{\mathsf{u}}
\newcommand{\ww}{\mathsf{w}}


\newcommand{\monomials}{\mathcal{X}}

%===Commitments===
\newcommand{\com}{\mathsf{Com}}
\newcommand{\keygen}{\mathsf{KeyGen}}
\newcommand{\abdlop}{\mathsf{ABDLOP}}
\newcommand{\extra}{\mathsf{ext}}
\newcommand{\transcript}{\mathsf{tr}}

\newcommand{\commitmentSpace}{\mathcal{M}}
\newcommand{\slackSpace}{\mathcal{S}}
\newcommand{\commitmentState}{\mathsf{st}}

%===Probabilities and Randomness===
%\newcommand{\rand}{\overset{\$}{\gets}}
\newcommand{\rand}{\gets}
\newcommand{\Prob}[1]{\Pr\left\lbrack{#1}\right\rbrack}
\newcommand{\CProb}[2]{\Pr\left\lbrack{#1}\,\middle|\,{#2}\right\rbrack} 
\newcommand{\stdev}{\mathfrak{s}}
\newcommand{\rnddist}{\chi}
\newcommand{\seed}{\mathsf{seed}}

%Payment System
\newcommand{\act}{\mathsf{act}}
\newcommand{\sn}{\mathsf{sn}}
\newcommand{\coin}{\mathsf{cn}}
\newcommand{\cnk}{\mathsf{cnk}}
\newcommand{\amt}{\mathsf{amt}}

\newcommand{\unum}[1]{^{\,(#1)}}

\newcommand{\SN}{\mathsf{SN}}
\newcommand{\ask}{\mathsf{ask}}
\newcommand{\ck}{\mathsf{ck}}


%sizes
\newcommand{\benchmarksize}{14.4}
\newcommand{\verencsize}{19.0}
\newcommand{\groupsigsize}{92}
\newcommand{\ciphertextsize}{1}

\newcommand{\raapprox}{16}
\newcommand{\rbapprox}{19}
\newcommand{\rcapprox}{20}
\newcommand{\rdapprox}{22}

%mults
\newcommand{\multZ}{\otimes}
\newcommand{\multR}{\underset{R}{\otimes}}
\newcommand{\multS}{\underset{S}{\otimes}}
\newcommand{\multSk}{\underset{S^k}{\otimes}}

\newcommand{\optcol}{black} %color for optional things in the protocol
\newcommand{\binone}{\bin{1}}
\newcommand{\bintwo}{\bin{2}}

\newcommand{\bvec}[1]{\bar{\bm #1}}
\newcommand{\bvecs}{\bar{\vecs}}
\newcommand{\bchal}{\bar{\chal}}

%
\newcommand{\bits}{\{0,1\}}
\newcommand{\frakd}{\mathfrak{d}}
\newcommand{\frakl}{\mathfrak{l}}
\newcommand{\diffIndex}{j}

%
% commands for proof sizes
\newcommand{\singlemonomialproof}{349.32}
\newcommand{\singlelargesetproof}{6.47}
\newcommand{\ACLMTimprovement}{10}
\newcommand{\NSimprovement}{NA}
\newcommand{\marlinsizes}{NA}
\newcommand{\marlinlargesetsmall}{14.66}
\newcommand{\marlinlargesetbig}{42.66}

%%% Local Variables:
%%% mode: latex
%%% TeX-master: "main"
%%% End:


\newcommand{\from}{\gets}
%%%%%%%%%%%%%%%%%%%%%%%%%%%%%%%%%%%%%%%%%%%%%%%%%%%%%%%%%%%%%%%%%%%%%%%%%%%%%%%%


%%%%%%%%%%%%%%%%%%%%%%%%%%%%%%%%%%%%%%%%%%%%%%%%%%%%%%%%%%%%%%%%%%%%%%%%%%%%%%%%
%%%%%%%%%%%%%%%%%%%%%%%%%%%%%%%%%%%%%%%%%%%%%%%%%%%%%%%%%%%%%%%%%%%%%%%%%%%%%%%%
%%%%%%%%%%%%%%%%%%%%%%%%%%%%%%%%%%%%%%%%%%%%%%%%%%%%%%%%%%%%%%%%%%%%%%%%%%%%%%%%
\begin{document}
	%%%%%%%%%%%%%%%%%%%%%%%%%%%%%%%%%%%%%%%%%%%%%%%%%%%%%%%%%%%%%%%%%%%%%%%%%%%%%%%%
	%%%%%%%%%%%%%%%%%%%%%%%%%%%%%%%%%%%%%%%%%%%%%%%%%%%%%%%%%%%%%%%%%%%%%%%%%%%%%%%%
	%%%%%%%%%%%%%%%%%%%%%%%%%%%%%%%%%%%%%%%%%%%%%%%%%%%%%%%%%%%%%%%%%%%%%%%%%%%%%%%%
	
	
	%%%%%%%%%%%%%%%%%%%%%%%%%%%%%%%%%%%%%%%%%%%%%%%%%%%%%%%%%%%%%%%%%%%%%%%%%%%%%%%%
	\title{
		More Efficient Aggregation of Falcon Signatures
	}
	%
	\author{
		\begin{tabular}[h!]{ccc}
		%	\FormatAuthor{Martin Albrecht}{martin.albrecht@kcl.ac.uk}{King's College London}
		%	\FormatAuthor{Alessandro Chiesa}{alessandro.chiesa@epfl.ch}{EPFL}
		%	\FormatAuthor{Giacomo Fenzi}{giacomo.fenzi@epfl.ch}{EPFL} 
		%	\FormatAuthor{Sasha Lapiha}{sasha.lapiha.2021@live.rhul.ac.uk}{Royal Holloway, University of London}
			\FormatAuthor{Ngoc Khanh Nguyen}{khanh.nguyen@epfl.ch}{EPFL}
		\end{tabular}
	}
    \institute{}
%	\date{\today}
	%%%%%%%%%%%%%%%%%%%%%%%%%%%%%%%%%%%%%%%%%%%%%%%%%%%%%%%%%%%%%%%%%%%%%%%%%%%%%%%%
	
	
	
	%%%%%%%%%%%%%%%%%%%%%%%%%%%%%%%%%%%%%%%%%%%%%%%%%%%%%%%%%%%%%%%%%%%%%%%%%%%%%%%%
	\maketitle
	%%%%%%%%%%%%%%%%%%%%%%%%%%%%%%%%%%%%%%%%%%%%%%%%%%%%%%%%%%%%%%%%%%%%%%%%%%%%%%%%
	
	
	
	%%%%%%%%%%%%%%%%%%%%%%%%%%%%%%%%%%%%%%%%%%%%%%%%%%%%%%%%%%%%%%%%%%%%%%%%%%%%%%%%
	%%%%%%%%%%%%%%%%%%%%%%%%%%%%%%%%%%%%%%%%%%%%%%%%%%%%%%%%%%%%%%%%%%%%%%%%%%%%%%%%
	%%%%%%%%%%%%%%%%%%%%%%%%%%%%%%%%%%%%%%%%%%%%%%%%%%%%%%%%%%%%%%%%%%%%%%%%%%%%%%%%
	\begin{abstract}
	
	\end{abstract}
	%%%%%%%%%%%%%%%%%%%%%%%%%%%%%%%%%%%%%%%%%%%%%%%%%%%%%%%%%%%%%%%%%%%%%%%%%%%%%%%%
	%%%%%%%%%%%%%%%%%%%%%%%%%%%%%%%%%%%%%%%%%%%%%%%%%%%%%%%%%%%%%%%%%%%%%%%%%%%%%%%%
	%%%%%%%%%%%%%%%%%%%%%%%%%%%%%%%%%%%%%%%%%%%%%%%%%%%%%%%%%%%%%%%%%%%%%%%%%%%%%%%%
	
	
	\iffull
	\thispagestyle{empty}
	\clearpage
	\setcounter{tocdepth}{2}
	\begin{spacing}{0.95}
		{\small\tableofcontents}
	\end{spacing}
	\thispagestyle{empty}
	\clearpage
	\setcounter{page}{1}
	\else
	%\setcounter{tocdepth}{1}
	%\begingroup
	%\let\clearpage\relax
	%\vspace{-1.8cm}
	%{\small\tableofcontents}
	%\endgroup
	\fi
	
	%%%%%%%%%%%%%%%%%%%%%%%%%%%%%%%%%%%%%%%%%%%%%%%%%%%%%%%%%%%%%%%%%%%%%%%%%%%%%%%%
	%%%%%%%%%%%%%%%%%%%%%%%%%%%%%%%%%%%%%%%%%%%%%%%%%%%%%%%%%%%%%%%%%%%%%%%%%%%%%%%%
	%%%%%%%%%%%%%%%%%%%%%%%%%%%%%%%%%%%%%%%%%%%%%%%%%%%%%%%%%%%%%%%%%%%%%%%%%%%%%%%%
	\doclearpage
	\section{Introduction}
	\label{sec:introduction}
	
	
	%%%%%%%%%%%%%%%%%%%%%%%%%%%%%%%%%%%%%%%%%%%%%%%%%%%%%%%%%%%%%%%%%%%%%%%%%%%%%%%%
	%%%%%%%%%%%%%%%%%%%%%%%%%%%%%%%%%%%%%%%%%%%%%%%%%%%%%%%%%%%%%%%%%%%%%%%%%%%%%%%%
	\subsection{Our results}
	\label{sec:our-results}
	
	
	%%%%%%%%%%%%%%%%%%%%%%%%%%%%%%%%%%%%%%%%%%%%%%%%%%%%%%%%%%%%%%%%%%%%%%%%%%%%%%%%
	%%%%%%%%%%%%%%%%%%%%%%%%%%%%%%%%%%%%%%%%%%%%%%%%%%%%%%%%%%%%%%%%%%%%%%%%%%%%%%%%
	\subsection{Related works}
	\label{sec:related-works}
	
	%%%%%%%%%%%%%%%%%%%%%%%%%%%%%%%%%%%%%%%%%%%%%%%%%%%%%%%%%%%%%%%%%%%%%%%%%%%%%%%%
	\subsubsection{A subsubsection}
	
	
	
	%%%%%%%%%%%%%%%%%%%%%%%%%%%%%%%%%%%%%%%%%%%%%%%%%%%%%%%%%%%%%%%%%%%%%%%%%%%%%%%%
	%%%%%%%%%%%%%%%%%%%%%%%%%%%%%%%%%%%%%%%%%%%%%%%%%%%%%%%%%%%%%%%%%%%%%%%%%%%%%%%%
	%%%%%%%%%%%%%%%%%%%%%%%%%%%%%%%%%%%%%%%%%%%%%%%%%%%%%%%%%%%%%%%%%%%%%%%%%%%%%%%%
	\doclearpage
	
	
	%%%%%%%%%%%%%%%%%%%%%%%%%%%%%%%%%%%%%%%%%%%%%%%%%%%%%%%%%%%%%%%%%%%%%%%%%%%%%%%%
	%%%%%%%%%%%%%%%%%%%%%%%%%%%%%%%%%%%%%%%%%%%%%%%%%%%%%%%%%%%%%%%%%%%%%%%%%%%%%%%%
	%%%%%%%%%%%%%%%%%%%%%%%%%%%%%%%%%%%%%%%%%%%%%%%%%%%%%%%%%%%%%%%%%%%%%%%%%%%%%%%%
	\doclearpage

\section{Preliminaries}\label{sec:prelims}
\paragraph{Notation.} We denote the security parameter by $\SecParam$, which is implicitly given to all algorithms unless specified otherwise. Further, we write $\negl$ (resp. $\poly$) to denote an unspecified negligible function (resp. polynomial) in $\SecParam$. In this work, we implicitly assume that the vast majority of the key parameters, e.g. the ring dimension, and the dimensions of matrices and vectors, are $\poly$. However, the modulus used in this work may be super-polynomial in $\SecParam$.

For $a, b \in \NN$ with $a < b$, write $[a, b] \coloneqq \{ a, a+1, \dots, b \}, [a] \coloneqq [1, a]$.
For $q \in \NN$ write $\ZZ_q$ for the integers modulo $q$.
We denote vectors with lowercase boldface (i.e. $\vecu, \vecv$) and matrices with uppercase boldface (i.e. $\matA, \matB$). For a vector $\vecx$ of length $n$, we write $x_i$ or $\vecx[i]$ for its $i$-th entry. Similarly, we define $\vecx_i := (x_1,\ldots,x_i)$ for $i\in [n]$. Given two vectors $\vecu,\vecv$, we denote by $(\vecu,\vecv)$ its concatenation. Also, $\varepsilon$ is an empty string.

\paragraph{Bits-to-integer conversion.} Let $k \geq 1$. For a vector $\vecb \in \ZZ^k_2$, we define the bits-to-integer conversion function $\integer(\vecb) := \sum^k_{i=1} b_i2^{i-1} \in [0,2^k-1]$. Clearly, if $\vecu \in \ZZ^k_2$ and $\vecv \in \ZZ^{l}_2$ then $\integer((\vecu,\vecv)) = \integer(\vecu) + 2^k\integer(\vecv)$.

\paragraph{Norms.}
We define the $\ell_p$ norm on $\CC^n$ as $\norm{\vecx}_p = \left(\sum_i |x_i|^p\right)^{1/p}$ for $p < \infty$ and $\infnorm{\vecx} \coloneqq \max_i |x_i|$. Unless otherwise specified, we use $\norm{\cdot}$ for the $\ell_2$ norm. We let the norm of a matrix be defined as the norm taken over the concatenation of columns of the matrix.

\paragraph{Lattices.} A subset $\Lattice \subseteq \RR^m$ is a lattice if: (i) $\veczero \in \Lattice$, and for $\vecx, \vecy \in \Lattice$, $\vecx + \vecy \in \Lattice$, and (ii) for every $\vecx \in \Lattice$, there exists $\epsilon > 0$ such that $\{ \vecy \in \RR^m : \norm{\vecx - \vecy} < \epsilon \} \cap \Lattice = \{ \vecx \}$.
We say $\matB \in \RR^{m \times k}$ is a basis for $\Lattice$ if its columns are linearly independent and $\Lattice = \genLattice(\matB) \coloneqq \{ \matB \vecz : \vecz \in \ZZ^k \}$. The span (as a vector space) of the basis of a lattice is the span of a lattice denoted as $\Span(\Lattice)$.
We also let $\Lattice^*$ be the dual lattice defined as $\Lattice^* = \{ \vecw \in \Span(\Lattice) :  \langle \Lattice, \vecw \rangle \subseteq \ZZ \}$.
We denote by $\tilde{\matB}$ the Gram-Schmidt orthonormalization of $\matB$. The Gram-Schmidt norm of $\matB$ is defined as $
 \|\tilde{\matB} \| \coloneqq  \max_{i \in [k]} \| \tilde{\vecb}_i\| $
where $\tilde{\vecb}_i$ is the $i$-th column of $\tilde{\matB}$.

\subsection{Power-of-Two Cyclotomic Rings}\label{subsec:cyclrings}
Let $N$ be a power-of-two and $\NumberField = \mathbb{Q}[X]/(X^N+1)$ be the $2N$-th cyclotomic field. Denote $\R =\ZZ[X]/(X^{N}+1)$ to be the ring of integers of $\NumberField$. For an odd prime $q$, we write $\Rq := \R/(q)$. We denote $\Rq^\times$ to be the set of invertible elements in $\Rq$. In this work, we pick $q \equiv 5 \pmod{8}$. Hence, we can  write 
\[X^N + 1 \equiv (X^{N/2}-r)(X^{N/2}+r) \pmod{q}\]
for a primitive $4^{\text{th}}$ root of unity $r \in \ZZ_q$ (i.e. $r^2 = -1$) where polynomials $X^{d/2}\pm r$ are irreducible modulo $q$. 

For $\matA \in \Rq^{n \times m}$, $\vecx \in \Rq^{m}$, we define the $q$-ary lattices (or lattice cosets) as  $\Lattice_\vecu^\perp(\matA) \coloneqq \{ \vecz \in \R^m : \matA \vecz = \vecu \bmod q \}$. We omit the subscript $\vecu$ if $\vecu = \veczero$. 

\paragraph{Invertibility results.} In this work, we consider a particular automorphism $\sigma : \R \rightarrow \R$ defined as $X \mapsto X^{-1}$. In the following, we prove that for any non-zero $c \in \Rq$, at least one of $c$ or $c + \sigma(c)$ is invertible over $\Rq$.

\begin{lemma}
Let $q \equiv 5 \pmod{8}$ and $c$ be a non-zero element of $\Rq$. Then, $\{c,c+\sigma(c)\} \cap \Rq^\times \neq \emptyset$.   
\end{lemma}
\begin{proof}
 First, note that $c+\sigma(c)$ is stable under $\sigma$. Hence, if $c+\sigma(c) \neq 0$ then by \cite[Lemma 2.6]{DBLP:conf/crypto/LyubashevskyNP22}, we get $c+ \sigma(c) \in \Rq^\times$. What we have left is the case $c + \sigma(c) = 0$. We prove that if $c \neq 0$ then $c$ must be invertible over $\Rq$.

 We follow the methodology from \cite[Lemma 2.6]{DBLP:conf/crypto/LyubashevskyNP22}. Let $c = \sum^{N-1}_{i=0} c_i X^i$. Then, $c + \sigma(c) = 0$ implies that
 \[ c = c_1X + c_2X^2 + \ldots + c_{N/2}X^{N/2} + c_{N/2-1}X^{N/2+1} + \ldots + c_1X^{N-1}.\]
 Now, note that
\[c \bmod (q,X^{N/2}\pm r) = \sum^{N/2-1}_{i=1} (c_i \mp rc_{N/2-i})X^i \mp rc_{N/2}.\]
Suppose $c \neq 0$. Then, one of the coefficients $c_1,\ldots,c_{N/2} \in \ZZ_q$ is non-zero, say $c_i$. First, if $i = N/2$ then the statement trivially holds. Now, assume $i \neq N/2$. We claim that for any sign $b\in \{-1,1\}$, either $c_i  - brc_{N/2-i}$ or $c_{N/2-i} - brc_{i}$ is not zero. Indeed, assume both of them were equal to zero, that is $c_i =  brc_{N/2-i}$ and $c_{N/2-i} = brc_{i}$ for $b \in \{-1,1\}$. Then $c_i = b rc_{N/2-i} = b^2r^2c_i =r^2c_i =  -c_i $ which is a contradiction since $c_i \neq 0$. Hence, we deduce that $c \bmod (p,X^{d/2}- r)$ and $c \bmod (p,X^{d/2} + r)$ are non-zero. The statement now follows by the Chinese Remainder Theorem.
\end{proof}
%\paragraph{Monomials.} We define $\monomials := \{ X^i : i \in \ZZ\}$ to be the set of monomials in $\R$. We use the result from \cite{DBLP:conf/asiacrypt/BenhamoudaCKLN14} which says that for any two distinct $x,y \in \monomials$, $\| 2/(x-y)\|_\infty = 1$.
%\paragraph{Short elements are invertible.} For $\kappa >0$, we define $S_\kappa := \{ x \in \Rq: \|x\|_\infty \leq \kappa \}$ to be the set of ring elements in $\Rq$ with infinity norm at most $\kappa$.  We recall the following invertibility result by Lyubashevsky and Seiler \cite{DBLP:conf/eurocrypt/LyubashevskyS18}.
%\begin{lemma}\label{lem:short-are-invertible}
% Let $1\leq l<N$ be a power-of-two and suppose $q \equiv 2N/l + 1 \pmod{4N/l}$. Then, every non-zero element in $S_\kappa$ is invertible over $\Rq$ as long as  $\kappa <\sqrt{l/N} \cdot q^{l/N}$.   
%\end{lemma}

\paragraph{Discrete Gaussian Distributions.} Let $\sigma > 0$ be a parameter and $\Lattice$ be a $m$-dimensional lattice. We then define the discrete Gaussian distribution $\discGauss_{\sigma, \vecc, \Lattice}$ over a lattice coset $\vecc + \Lattice$ as follows.
\begin{equation*}
  \rho_{\sigma, \vecc}(\vecz) \coloneqq \exp\left( - \frac{\pi \norm{\vecz - \vecc}^2}{\sigma^2} \right) \text{ and } \discGauss_{\sigma, \vecc, \Lattice}(\vecz) \coloneqq \frac{\rho_{\sigma, \vecc}(\vecz)}{\sum_{\vecx \in \Lattice} \rho_{\sigma, \vecc}(\vecx)}
\end{equation*}
When $\vecc = \veczero$ or $\Lattice = \ZZ^m$, we will omit it from the notation. We naturally extend this notion for lattices over the ring of integers $\R$, and for matrices by sampling column-wise.

Let $\matA \in \Rq^{n \times m}$ be a matrix over $\Rq$ and take any $\vecu \in \Rq^n$. We write $\vecs \from \matA^{-1}_\sigma (\vecu)$ to denote sampling $\vecs \from \discGauss^{mN}_{\sigma}$ conditioned on $\matA\vecs = \vecu$. 

\paragraph{Smoothing parameter.}
The smoothing parameter $\smoothing_\epsilon(\Lattice)$ of a lattice is the smallest $s > 0$ such that $\rho_{1/s}(\Lattice^*) \leq 1 + \epsilon$. Below we recall the standard upper-bound on the smoothing parameter \cite{MicciancioR07,DBLP:conf/stoc/GentryPV08}.

\begin{lemma}
  \label{lemma:smoothing}
  Let $\Lattice \subseteq \RR^m$ be a lattice, and let $\epsilon > 0$.
  Then, for every basis $\matB$ of $\Lattice$,
  \begin{equation*}
    \smoothing_{\epsilon}(\Lattice) \leq \tilde{\norm{\matB}} \cdot \sqrt{\frac{\ln(2m(1 + 1/\epsilon))}{\pi}} \enspace.
  \end{equation*}
\end{lemma}

\paragraph{Rejection sampling.} We recall the generalised version introduced recently by Boschini et al.\ \cite{DBLP:conf/crypto/BoschiniTT22} for discrete Gaussian over arbitrary lattices (here we skip the general case for ellipsoidal Gaussians).
\begin{figure}[t]
				\centering
				\resizebox{\textwidth}{!}{
				\fbox{
					\begin{minipage}[t]{0.59\textwidth}
						\underline{ $\mathsf{RejSamp}$:}
						\begin{algorithmic}[1]
							\State $(\vecu,\vecv) \gets h$
							\State $\vecz \gets \discGauss^{mN}_{\sigma, \vecv+\vecu,\Lattice}$
							\State $\pcreturn (\vecu, \vecv,\vecz)$ with prob. $\min\left(\frac{ \discGauss^m_{\sigma}(\vecz)}{M \cdot  \discGauss^m_{\sigma, \vecv}(\vecz)},1\right)$
						\end{algorithmic}
					\end{minipage}
					\quad
					\begin{minipage}[t]{0.4\textwidth}
						\underline{ $\mathsf{SimRS}$:}
						\begin{algorithmic}[1]
							\State $(\vecu,\vecv) \gets h$
							\State $\vecz \gets \discGauss^{mN}_{\sigma,\vecu,\Lattice}$
							\item $\pcreturn (\vecu, \vecv,\vecz)$ with prob. $\frac{1}{M}$
						\end{algorithmic}
					\end{minipage}
				}}
				\caption{Rejection sampling \cite{DBLP:conf/crypto/BoschiniTT22}.}
				\label{fig:rejsamp}
			\end{figure}
			%
\begin{lemma}[Rejection Sampling \cite{DBLP:conf/crypto/BoschiniTT22}]
	\label{lem:rejsamp}
	Take any $\alpha, T>0$ and $\varepsilon \leq 1/2$. Let $\Lattice \subseteq \R^m$ be a lattice over $\R$ and $\sigma \geq \max(\alpha T, \smoothing_{\varepsilon}(\Lattice))$ be a parameter. Let $h: \R^m \times \R^m \rightarrow [0,1]$ be a probability distribution which returns $(\vecu,\vecv)$ where the vector $\vecv$ satisfies $\|\vecv\| \leq T$\footnote{One may think of $\vecu$ (resp. $\vecv$) as the public (resp. private) shift.}. Further, define $M \coloneqq  \exp(\frac{\pi}{\alpha^2} + 1)$ and $\epsilon \coloneqq  2\frac{1+\varepsilon}{1-\varepsilon}\exp(-\alpha^2 \cdot \frac{\pi-1}{\pi^2})$. Then, the statistical distance between distributions $\mathsf{RejSamp}$ and $\mathsf{SimRS}$ defined in \Cref{fig:rejsamp} is at most $\frac{\epsilon}{2M} + \frac{2\varepsilon}{M}$. Moreover, the probability that $\mathsf{RejSamp}$ outputs something is at least $\frac{1-\epsilon}{M}\left(1-\frac{4\varepsilon}{(1+\varepsilon)^2}\right)$.
\end{lemma}

\paragraph{Module-SIS.} We recall the standard Module-SIS \cite{DBLP:journals/dcc/LangloisS15} assumption.

\begin{definition}[Module-SIS]\label{def:msis}
	 Let $q = q(\secpar)$, $n = n(\secpar)$, $m = m(\secpar)$, $\beta = \beta(\secpar)$ and $N = N(\secpar)$. We say that the $\msis_{n,m,N,q,\beta}$ assumption holds if for any PPT adversary $\adversary{A}$, the following holds:
	\begin{equation*}
		\CProb{
			\begin{array}{c}
				\matA \vecs = \veczero \land 
				0 < \norm{\vecx} \leq \beta
			\end{array}
		}
		{ 
			\begin{array}{c}
				\matA\from \Rq^{n \times m}\\
				\vecs \from \adversary{A}(\matA)
			\end{array}
		} \leq \negl \enspace.
	\end{equation*}
\end{definition}
\subsection{Trapdoor Sampling}

\paragraph{Gadget matrix.} Let $\delta \geq 2$. We set $\q \coloneqq \lfloor \log_\delta q \rfloor + 1$, and $\vecg^\top = [1, \delta, \dots, \delta^{\q - 1}] \in \R_q^{1 \times \q}$ and $\matG_n \coloneqq \matI_n \otimes \vecg^\top \in \R_q^{n \times n\q}$. When the dimension are clear from context we simply write $\matG$.
  Write $\matG^{-1}_n: \Rq^{n \times t} \to \Rq^{n\q \times t}$ for the inverse function that takes a matrix of entries in $\Rq$, and  decomposes each entry w.r.t. the base $\delta$.
  %By construction $\matG_n \cdot \matG^{-1}_n(\matA) = \matA$ for $\matA \in \Rq^{n \times t}$. 
  We also write $\vecg^{-1}$ for $\matG_1^{-1}$.

Next, we recall the trapdoor generation from \cite{DBLP:conf/eurocrypt/MicciancioP12,FenNgu23}.
\begin{lemma}[Trapdoor Generation]
  \label{lemma:genTrapdoor}
  Let $N,n>0, t = n\q$ and $\matG_n \in \Rq^{n \times t}$ be the gadget matrix. Take $ m \geq t + n + \omega(\log \SecParam)$. Then, there is a PPT algorithm $\trapgen(n, m)$ that with an overwhelming probability returns two matrices $(\matA, \matR) \in \Rq^{n \times m} \times \Rq^{m \times t}$ such that $\matA \matR = \matG_n$ and  $\|\matR\| \leq  \stdev\sqrt{2t(m-t)N}$ where $\stdev > 2N \cdot q^{\frac{n}{m-t} + \frac{2}{N(m-t)}}$. Moreover, $\matA$ is statistically close to a uniformly random matrix in $\Rq^{n \times m}$.
\end{lemma}

\noindent The next lemma \cite{DBLP:conf/eurocrypt/MicciancioP12} states that given a short $\matG$-trapdoor matrix $\matR$ for $\matA$, one can efficiently sample preimages of $\matA$ according to the discrete Gaussian distribution. We further merge the result with the tail-bound inequality from \cite{MicciancioR07}.

\begin{lemma}[Preimage Sampling]\label{lem:samplepre}
Let $N,n,m>0$, $t = n\q$ and $k = \max(n,m)$. Then, there exists a PPT algorithm $\samplepre(\matA,\matR,\vecv,\sigma)$ that takes as input a matrix $\matA \in \Rq^{n \times m}$, a $\matG_n$-trapdoor $\matR \in \Rq^{m \times t}$ for $\matA$ with a tag $\matH$, a target vector $\vecv \in \Rq^n$ in the column-span of $\matA$, and a Gaussian parameter $\sigma$, and outputs a vector $\vecs \in \Rq^m$ such that $\matA\vecs=\vecv$. Further, if $\sigma \geq \delta \|\matR\| \cdot \omega(N\sqrt{\log kN})$, then the statistical distance between the following distributions is negligible:
  \begin{equation*}
	    \left\{ \vecs \from \samplepre(\matA,\matR,\vecv,\sigma) \right\} \text{ and }
	    \left\{ \vecs \from \matA^{-1}_\sigma(\vecv)
	    \right\}
	  \end{equation*}
and in particular, $\Pr[\|\vecs\| > \sigma \sqrt{mN} :  \vecs \from \samplepre(\matA,\matR,\vecv,\sigma)]$ is negligible.

We extend this algorithm for matrices, i.e. for a matrix $\matV \in \Rq^{n \times \ell}$ with columns $\vecv_1,\ldots,\vecv_\ell$, we define $\samplepre(\matA,\matR,\matV,\sigma)$ to be the algorithm which returns a matrix $\matS \in \Rq^{m \times \ell}$, where the $i$-th column is the output of $\samplepre(\matA,\matR,\vecv_i,\sigma)$.
\end{lemma}





\subsection{Falcon Signatures}\label{sec:ntru-lattices}
The NTRU lattice associated to $h \in \Rq$ is defined as
			\begin{equation*}
				\Lattice_{h} \coloneqq  \{(u,v) \in \R^2 : u + vh = 0 \bmod q \}
				\enspace.
			\end{equation*}
			%
%			Then, $\Lattice_{h}$ is a $2N$-dimensional full-rank lattice generated by the rows of \begin{equation*}
%A_{h,q} \coloneqq  \begin{bmatrix} -\rot(h) & \matI_N \\ q \cdot \matI_N & 0 \end{bmatrix} \in \ZZ^{2N \times 2N} \enspace.
%\end{equation*}
			We use the main result of Stehlé  and Steinfeld \cite{DBLP:journals/iacr/StehleS13}, which says that there is an efficient algorithm that outputs $(h,\NTRUbasis)$ such that $h$ is statistically close to a uniform distribution over $\Rq^\times$ and $\NTRUbasis$ is a basis for $\Lattice_{h}$. 
			%
			\begin{lemma}[NTRU Trapdoor Generation \cite{DBLP:journals/iacr/StehleS13}]\label{lemma:NTRUTrapGen}
                Let $q = \omega(N)$ such that $q \equiv 1 \pmod{2N}$. Take $\epsilon \in (0,1/3)$ and $\stdev \geq \max(N\sqrt{\ln(8Nq)}\cdot q^{1/2 + \epsilon}, \omega(N^{3/2}\ln^{3/2}N))$. Then, there is a PPT algorithm $\NTRUTrapGen(q,N,\stdev)$ which with an overwhelming probability outputs $h \in \Rq$ and a basis $\NTRUbasis$ of $\Lattice_{h}$ such that $\|\tilde{\matT}_\ntru\| \leq  N\stdev$. Furthermore, the statistical distance between the distribution of $h$ and uniform over $\Rq^\times$ is at most $2^{10N}q^{-\lfloor \epsilon N \rfloor}$.
			\end{lemma}
			%
%			\noindent The short basis can now be used for preimage sampling using the GPV framework \cite{DBLP:conf/stoc/GentryPV08}.
			%
%			\begin{lemma}[NTRU Preimage Sampling \cite{DBLP:conf/stoc/GentryPV08,DBLP:journals/iacr/StehleS13}]
%				\label{lem:gaussian_sampler}
%				Define $q,N,\stdev$ as in \Cref{lemma:NTRUTrapGen}, and $\sigma  \geq N^{3/2} \stdev \omega(\sqrt{\log N})$. Then, there is a PPT algorithm $\GaussianSampler$, that takes $(h,\NTRUbasis) \rand \NTRUTrapGen(q,N,\stdev)$, a target vector $c \in \Rq$ and a parameter $\sigma >0$  as input, and outputs a pair $(u,v)\in \Rq^{2}$ such that \begin{equation*}
%\Delta\left(\begin{bmatrix} h &  1 \end{bmatrix}^{-1}_\sigma(c), \GaussianSampler(h,\NTRUbasis,c,\sigma)\right) \leq \negl \enspace.
%\end{equation*}
%			\end{lemma}


\subsection{Commitment Scheme}
We recall the notion of a commitment scheme with relaxed binding.
\begin{definition}
  Let $\commitmentScheme = (\setup, \commit, \open)$ be a triple of PPT algorithms. We say that $\commitmentScheme$ is a commitment scheme over $\commitmentSpace$ with slack space $\slackSpace$ if it has the following syntax:
  \begin{itemize}[nolistsep]
    \item $\setup(\secparam) \to \crs$ takes a security parameter $\secpar$ (specified in unary) and outputs a common reference string $\crs$.
    \item $\commit(\crs, m) \to (C, \commitmentState)$ takes a common reference string $\crs$ a message $m \in \commitmentSpace$ and outputs a commitment $C$ and decommitment state $\commitmentState$.
    \item $\open(\crs, C, m, \commitmentState, c)$ takes a common reference string $\crs$, a commitment $C$, a message $m \in \commitmentSpace$, a decommitment state $\commitmentState$ and a relaxation factor \footnote{We implicitly assume that if $c \not\in \slackSpace$ then $\open$ automatically returns $0$.} $c \in \slackSpace$ and outputs a bit indicating whether $C$ is a valid commitment to $m$ under $\crs$.
  \end{itemize}
\end{definition}
\noindent We define the key properties of the commitment scheme: correctness, (relaxed) binding and hiding. In the following, we denote the message space as $\commitmentSpace$ and the slack space as $\slackSpace$. 
\begin{definition}[Completeness] A commitment scheme $\commitmentScheme = (\setup, \commit, \open)$ satisfies completeness if there exists a global relaxation factor $c^* \in \slackSpace$ such that for every $m \in \commitmentSpace$:
      \begin{equation*}
        \Pr\left[
          \open(\crs, C, m, \commitmentState, c^*) = 1
          \middle|
          \begin{array}{c}
            \crs \from \setup(\secparam) \\
            C, \commitmentState \from \commit(\crs, m) \\
          \end{array}
        \right] \geq 1 - \negl.
      \end{equation*}
\end{definition}
\noindent For the notion of relaxed binding, we assume that the adversary comes up with two different openings with \textit{the same} relaxation factor.
\begin{definition}[Relaxed Binding]
     A commitment scheme $\commitmentScheme = (\setup, \commit, \open)$ satisfies relaxed binding if for every PPT adversary $\adversary{A}$:
      \begin{equation*}
      \scalemath{0.95}{
        \Pr\left[
          \begin{array}{c}
            m \neq m' \land m,m' \in \commitmentSpace \land \\
            \open(\crs, C, m, \polystate, c) = \open(\crs, C, m', \polystate', c) = 1
          \end{array}
        \middle|
          \begin{array}{c}
            \crs \from \setup(\secparam) \\
            (C, (m, \polystate), (m, \polystate'), c) \from \adversary{A}(\crs) \\
          \end{array}
        \right] = \negl.}
      \end{equation*}

\end{definition}
\begin{definition}[Hiding] A commitment scheme $\commitmentScheme = (\setup, \commit, \open)$ satisfies hiding if for every (stateful) PPT adversary $\adversary{A}$:
      \begin{equation*}
        \Pr\left[
          b' = b
          \middle|
          \begin{array}{c}
            \crs \from \setup(\secparam), (m_0,m_1) \from \Adversary(\crs) \\
            b \from \{0,1\} \\
            C, \commitmentState \from \commit(\crs, m_b) \\
            b' \from \Adversary(C)
          \end{array}
        \right] \leq \frac{1}{2} + \negl.
      \end{equation*}
\end{definition}
\subsubsection{BASIS Commitment Scheme}
In this section we recall a compressing commitment scheme which stems from the vector commitment construction of Wee and Wu \cite{WeeWu23}. We inherit a crucial property from the aforementioned work that we support committing to arbitrarily large ring elements. Let $\ell$ be the length of the committed vectors over $\Rq$.  Finally, we define $t = n\q$ and $\matG \coloneqq  \matG_n \in \Rq^{n \times t}$. The commitment scheme is described in \Cref{fig:basis-commitment}.

\begin{figure}
    \centering
    \begin{theo}[$\mathsf{BASIS}$ Commitment Scheme]
    \underline{$\polysetup(1^\SecParam)$}
    \begin{enumerate}[noitemsep]
      \item Sample $(\matA, \matR) \from \trapgen(n, m)$.
      \item Sample $\matW_1,\ldots,\matW_\ell \from \GL(n,\Rq)$
      \item Let $\matR_i \coloneqq \matR \matG^{-1}(\matW_i^{-1} \matG)$ for $i \in [0,d]$.
      \item Set \begin{equation*} 
      \matB \coloneqq 
      \left[\begin{array}{@{}ccc|c@{}}
        \matW_1\matA &        & & -\matG \\ 
        & \ddots & & \vdots \\ 
        & & \matW_{\ell} \matA & -\matG
      \end{array}\right], \quad 
      \tilde{\matR} \coloneqq
      \begin{bmatrix}
        \matR_{1} & & \\
        & \ddots & \\
        & & \matR_{\ell} \\
        \hline 
        & \matzero &
      \end{bmatrix}
      \enspace.
    \end{equation*}
    \item Sample $\matT \from \samplepre(\matB, \tilde{\matR}, \matG_{n\ell}, \sigma_0)$.
    \item Return $\crs \coloneqq (\matA, \matW_1,\ldots,\matW_\ell, \matT)$.
    \end{enumerate}
    \vspace{5mm}
    $\underline{\polycommit(\crs, \vecf \in \Rq^{\ell})}$
    \begin{enumerate}[noitemsep]
    \item Parse $\vecf \coloneqq  (f_1,\ldots,f_d)$
      \item Set $\matU \coloneqq \begin{bmatrix} -f_0 \matW_1 \matG \\ \vdots \\ -f_\ell \matW_\ell \matG \end{bmatrix}$
      \item Sample $\begin{bmatrix} \matS_1 \\ \vdots \\ \matS_\ell \\ \hat{\matC} \end{bmatrix} \from \samplepre(\matB, \matU, \matT, \sigma_1)$.
      \item Set $\matC \coloneqq \matG \hat{\matC}$.
      \item Return $(\matC, \polystate \coloneqq (\matS_i)_{i \in [\ell]})$.
    \end{enumerate}
    \vspace{5mm}
    $\underline{\polyopen(\crs, \matC, \vecf \in \Rq^{\ell}, \polystate)}$
    \begin{enumerate}[noitemsep]
    \item Parse $\vecf \coloneqq  (f_1,\ldots,f_\ell)$ .
      \item Parse $\polystate \coloneqq (\matS_i)_{i \in [\ell]}$.
      \item Return $1$ if and only if for all $i \in [\ell]$, 
      \begin{itemize}
          \item $\matA \matS_i + f_i \matG = \matW_i^{-1} \matC$.
          \item $\|\matS_i\| \leq \gamma$.
      \end{itemize}
    \end{enumerate}
    \end{theo}
    \caption{$\mathsf{BASIS}$ commitment scheme for arbitrary messages in the message space $\commitmentSpace = \Rq^{\ell}$. Here, $\matG \in \Rq^{n \times n\q}$ is the gadget matrix of height $n$.}
    \label{fig:basis-commitment}
\end{figure}

\section{Aggregation of Falcon Signatures}
In this section we show how to publicly aggregate Falcon signatures using the $\mathsf{BASIS}$ commitment scheme. Namely, suppose we are given $\ell$ signatures $\{(u_i,v_i)\}_{i \in [\ell]} \in \Rq^2$ on distinct messages $m_1,\ldots,m_\ell$ corresponding to the verification keys $a_1,\ldots,a_\ell \in \Rq$ such that:
\[ a_iu_i + v_i = H(m_i) \quad \text{and} \quad \|(u_i,v_i)\| \leq \beta \quad \text{for } i \in [\ell].\]
Intuitively, we will provide a succinct proof of knowledge of $\{(u_i,v_i)\}_{i \in [k]} \in \Rq^2$ which satisfy the relation above.  Let $\crs = (\matA,\matW_1,\ldots,\matW_\ell,\matT)$ be the public parameters related to the $\mathsf{BASIS}$ commitment. Denote $\bar{\matA} := \sigma(\matA)$, and similarly for all the other matrices included in $\crs$. Then, define $\bar{\crs} := (\bar{\matA},\bar{\matW}_1,\ldots,\bar{\matW}_\ell,\bar{\matT})$.

We first compute the following commitments:
\begin{align*}
& \left(\matC_a, (\matS_i)_{i \in [\ell]}\right) \from \commit(\crs,(a_1,\ldots,a_\ell)), \quad \left(\bar{\matC}_a, (\bar{\matS}_i)_{i \in [\ell]}\right) \from \commit(\bar{\crs},(a_1,\ldots,a_\ell)) \\
& \left(\matC_u, (\matU_i)_{i \in [\ell]}\right) \from \commit(\crs,(u_1,\ldots,u_\ell)), \quad \left(\bar{\matC}_u, (\bar{\matU}_i)_{i \in [\ell]}\right) \from \commit(\bar{\crs},(u_1,\ldots,u_\ell))\\
& \left(\matC_v, (\matV_i)_{i \in [\ell]}\right) \from \commit(\crs,(v_1,\ldots,v_\ell)), \quad \left(\bar{\matC}_v, (\bar{\matV}_i)_{i \in [\ell]}\right) \from \commit(\bar{\crs},(v_1,\ldots,v_\ell)).
\end{align*}
By construction of the $\mathsf{BASIS}$ commitment scheme, we have for $i \in [\ell]$:
\begin{align*}
& \matA\matS_i + a_i\matG = \matW_i^{-1} \matC_a,  \quad  \matA\matU_i + u_i\matG = \matW_i^{-1} \matC_u \quad \text{and} \quad \matA\matV_i + v_i\matG = \matW_i^{-1} \matC_v.   
\end{align*}
In particular,
\[ \matW_i^{-1}\left(\matC_u\matG^{-1}(\matW_i^{-1}\matC_a) + \matC_v\right) = \matA\left(\matU_i\matG^{-1}(\matW_i^{-1}\matC_a) + u_i\matS_i + \matV_i \right) + (a_iu_i + v_i)\matG.\]
Now, we call the random oracle $\ro$ to obtain challenges:
\begin{equation}\label{eq:fiat-shamir} (\alpha_1,\ldots,\alpha_\ell) := \ro(\crs, (a_1,\ldots,a_\ell), (m_1,\ldots,m_\ell), (\matC_a,\bar{\matC}_a,\matC_u,\bar{\matC}_u,\matC_v,\bar{\matC}_v))).\end{equation}
We then aggregate the commitment openings as follows:
\begin{align*}
& \matZ := \sum^\ell_{i=1} \alpha_i\matS_i, \quad   \matX := \sum^\ell_{i=1} \alpha_i\matU_i, \quad \matY := \sum^\ell_{i=1} \alpha_i\matV_i, \quad \matR := \sum^\ell_{i=1} \alpha_i\left(\matU_i\matG^{-1}(\matW_i^{-1}\matC_a) + u_i\matS_i + \matV_i \right) \\
& \bar{\matZ} := \sum^\ell_{i=1} \alpha_i\bar{\matS}_i, \quad   \bar{\matX} := \sum^\ell_{i=1} \alpha_i\bar{\matU}_i, \quad \bar{\matY} := \sum^\ell_{i=1} \alpha_i\bar{\matV}_i, \quad \bar{\matR} := \sum^\ell_{i=1} \alpha_i\left(\bar{\matU}_i\matG^{-1}(\matW_i^{-1}\bar{\matC}_a) + u_i\bar{\matS}_i + \bar{\matV}_i \right).
\end{align*}
Further,
\[ x := \sum^\ell_{i=1} \alpha_iu_i, \quad y := \sum^\ell_{i=1} \alpha_iv_i.\]
The aggregated signature is
$\mathsf{s} := ((\matC_a,\bar{\matC}_a,\matC_u,\bar{\matC}_u,\matC_v,\bar{\matC}_v),(\matX,\bar{\matX},\matY,\bar{\matY},\matZ,\bar{\matZ},\matR,\bar{\matR}),(x,y))$.

The aggregation verifier is given  public parameters $\crs$, verification keys $a_1,\ldots,a_\ell$, messages $m_1,\ldots,m_\ell$ and the aggregated signature $\mathsf{s}$. It first recomputes the challenges $(\alpha_1,\ldots,\alpha_\ell)$ as in \eqref{eq:fiat-shamir}. Then, it accepts if all the following conditions hold:
\begin{align*} & \matA\matZ + \left(\sum^\ell_{i=1} \alpha_i a_i\right) \matG = \left(\sum^{\ell}_{i=1} \alpha_i\matW^{-1}_i\right) \matC_a, \quad \bar{\matA}\bar{\matZ} + \left(\sum^\ell_{i=1} \alpha_i a_i\right) \matG = \left(\sum^{\ell}_{i=1} \alpha_i\matW^{-1}_i\right) \bar{\matC}_a, \\
& \matA\matX + x \matG = \left(\sum^{\ell}_{i=1} \alpha_i\matW^{-1}_i\right) \matC_u, \quad \bar{\matA}\bar{\matX} + x \matG = \left(\sum^{\ell}_{i=1} \alpha_i\bar{\matW}^{-1}_i\right) \bar{\matC}_u, \\
&  \matA\matY + y \matG = \left(\sum^{\ell}_{i=1} \alpha_i\matW^{-1}_i\right) \matC_v,  \quad \bar{\matA}\bar{\matY} + y \matG = \left(\sum^{\ell}_{i=1} \alpha_i\bar{\matW}^{-1}_i\right) \bar{\matC}_v, \\
& \matA \matR +  \left(\sum^\ell_{i=1} \alpha_i H(m_i) \right) \matG = \sum^\ell_{i=1}\alpha_i\matW_i^{-1}\left(\matC_u\matG^{-1}(\matW_i^{-1}\matC_a) + \matC_v\right), \\
& \bar{\matA} \bar{\matR} +  \left(\sum^\ell_{i=1} \alpha_i H(m_i) \right) \matG = \sum^\ell_{i=1}\alpha_i\bar{\matW}_i^{-1}\left(\bar{\matC}_u\matG^{-1}(\bar{\matW}_i^{-1}\bar{\matC}_a) + \bar{\matC}_v\right), \\
& \|\matZ\|, \| \bar{\matZ}\| \leq \beta_Z, \|\matX\|,\| \bar{\matX}\| \leq \beta_X, \|\matY\|,\| \bar{\matY}\| \leq \beta_Y, \|\matR\|,\| \bar{\matR}\| \leq \beta_R, \\
& \|(x,y)\| \leq \beta_{xy}.
\end{align*}

%%%%%%%%%%%%%%%%%%%%%%%%%%%%%%%%%%%%%%%%%%%%%%%%%%%%%%%%%%%%%%%%%%%%%%%%%%%%%%%%
%%%%%%%%%%%%%%%%%%%%%%%%%%%%%%%%%%%%%%%%%%%%%%%%%%%%%%%%%%%%%%%%%%%%%%%%%%%%%%%%
%%%%%%%%%%%%%%%%%%%%%%%%%%%%%%%%%%%%%%%%%%%%%%%%%%%%%%%%%%%%%%%%%%%%%%%%%%%%%%%%
\doclearpage

%%%%%%%%%%%%%%%%%%%%%%%%%%%%%%%%%%%%%%%%%%%%%%%%%%%%%%%%%%%%%%%%%%%%%%%%%%%%%%%%
%%%%%%%%%%%%%%%%%%%%%%%%%%%%%%%%%%%%%%%%%%%%%%%%%%%%%%%%%%%%%%%%%%%%%%%%%%%%%%%%
%%%%%%%%%%%%%%%%%%%%%%%%%%%%%%%%%%%%%%%%%%%%%%%%%%%%%%%%%%%%%%%%%%%%%%%%%%%%%%%%

%%%%%%%%%%%%%%%%%%%%%%%%%%%%%%%%%%%%%%%%%%%%%%%%%%%%%%%%%%%%%%%%%%%%%%%%%%%%%%%%
\appendix
%%%%%%%%%%%%%%%%%%%%%%%%%%%%%%%%%%%%%%%%%%%%%%%%%%%%%%%%%%%%%%%%%%%%%%%%%%%%%%%%


%%%%%%%%%%%%%%%%%%%%%%%%%%%%%%%%%%%%%%%%%%%%%%%%%%%%%%%%%%%%%%%%%%%%%%%%%%%%%%%%
%%%%%%%%%%%%%%%%%%%%%%%%%%%%%%%%%%%%%%%%%%%%%%%%%%%%%%%%%%%%%%%%%%%%%%%%%%%%%%%%
%%%%%%%%%%%%%%%%%%%%%%%%%%%%%%%%%%%%%%%%%%%%%%%%%%%%%%%%%%%%%%%%%%%%%%%%%%%%%%%%
\clearpage
%\iffull
\section*{Acknowledgments}\label{sec:acknowledgements}
Placeholder

%%%%%%%%%%%%%%%%%%%%%%%%%%%%%%%%%%%%%%%%%%%%%%%%%%%%%%%%%%%%%%%%%%%%%%%%%%%%%%%%
%%%%%%%%%%%%%%%%%%%%%%%%%%%%%%%%%%%%%%%%%%%%%%%%%%%%%%%%%%%%%%%%%%%%%%%%%%%%%%%%
%%%%%%%%%%%%%%%%%%%%%%%%%%%%%%%%%%%%%%%%%%%%%%%%%%%%%%%%%%%%%%%%%%%%%%%%%%%%%%%%
%\fi
%\iffull
\printbibliography
%\fi
%%%%%%%%%%%%%%%%%%%%%%%%%%%%%%%%%%%%%%%%%%%%%%%%%%%%%%%%%%%%%%%%%%%%%%%%%%%%%%%%
%%%%%%%%%%%%%%%%%%%%%%%%%%%%%%%%%%%%%%%%%%%%%%%%%%%%%%%%%%%%%%%%%%%%%%%%%%%%%%%%
%%%%%%%%%%%%%%%%%%%%%%%%%%%%%%%%%%%%%%%%%%%%%%%%%%%%%%%%%%%%%%%%%%%%%%%%%%%%%%%%
\end{document}

%%%%%%%%%%%%%%%%%%%%%%%%%%%%%%%%%%%%%%%%%%%%%%%%%%%%%%%%%%%%%%%%%%%%%%%%%%%%%%%%
%%%%%%%%%%%%%%%%%%%%%%%%%%%%%%%%%%%%%%%%%%%%%%%%%%%%%%%%%%%%%%%%%%%%%%%%%%%%%%%%
%%%%%%%%%%%%%%%%%%%%%%%%%%%%%%%%%%%%%%%%%%%%%%%%%%%%%%%%%%%%%%%%%%%%%%%%%%%%%%%%
%%% Local Variables:
%%% mode: latex
%%% TeX-master: t
%%% End:
